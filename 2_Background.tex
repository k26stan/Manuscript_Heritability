%% BACKGROUND %%

\section*{Background}

Rheumatoid arthritis (RA) is an autoimmune disorder characterized by stiff and painful joints, chronic inflammation, synovitis, irreparable joint damage, and the presence of auto-antibodies. The current diagnostic approach incorporates information concerning synovitis, serology, acute-phase reaction, and symptom duration \cite{aletaha_2010_2010,radner_performance_2014}. While the precise etiology is unclear, there is a strong genetic component as concordance in monozygotic twins is 15-30\% while the population prevalence is around 1\%. Some have estimated the heritability for the disease to be as high as 50\% \cite{perricone_overview_2011,mcinnes_pathogenesis_2011}. Despite an unknown etiology, these signature features have allowed standardized diagnosis and assessment of RA. One commonly used method for assessing disease severity is the Modified Disease Activity Score for twenty-eight joints (DAS). DAS incorporates the swollen and tender joints counts (SJC and TJC, respectively) out of 28 joints, the erythrocyte sedimentation rate (ESR) and a visual analog scale (VAS) score for general health in a formula where a higher score (up to 10) indicates a more severe disease state \cite{van_der_heijde_judging_1990,prevoo_modified_1995}. Variations of this score include the use of C-Reactive Protein (CRP) as an inflammation marker to replace ESR \cite{skogh_twenty_2003,fransen_disease_2003}. \textcolor{green}{Furthermore, categorical definitions of response have been developed by the American College of Rheumatology (ACR) and the European League Against Rheumatism (EULAR). ACR20 and ACR50 represent a 20\% or 50\% improvement of disease state based on a combination of SJC, TJC, patient and physician global assessments, pain, disability, and an acute-phase reactant \cite{felson_american_1995,felson_american_2011}. EULAR classified patients' response into good-, moderate-, no-response based on their change in DAS and final DAS after treatment \cite{fransen_disease_2005}. Change in DAS and the categorical response phenotypes (e.g., EULAR, ACR50) are commonly used response metrics, but other measurements may have value in pharmacogenetic studies as well.}

Early recognition and effective intervention can slow the progression, minimize joint destruction, and promote remission \cite{finckh_long-term_2006}. \textcolor{green}{Disease-modifying anti-rheumatic drugs (DMARDs) are widely used} in patients at all stages of RA in order to slow the progression of the disease \cite{singh_2012_2012,smolen_eular_2014,smolen_eular_2010}. While various DMARDs are available, methotrexate (MTX) has been shown to be safe and effective for long-term use in many cases; however, MTX proves ineffective or the treatment is unsustainable due to adverse events in over half of patients \cite{pincus_methotrexate_2003,pincus_update_2013,smolen_treating_2015,odell_therapies_2013,goekoop-ruiterman_clinical_2005,moreland_randomized_2012}. Anti-TNF biologic agents are often added to MTX as the next course of action. Since the first anti-TNF agent was approved for treatment of RA in 1998, they have successfully improved RA disease state in countless patients; however, these treatments are not effective in all patients \cite{firestein_evolving_2003}. Consequently, many studies have attempted to identify biomarkers that predict efficacy of a particular DMARD or class of DMARDs, including anti-TNF agents.
% such as adalimumab (Humira®), infliximab (Remicade®), certolizumab pegol (Cimzia®), etanercept (Enbrel®), or golimumab (Simponi®)

\textcolor{green}{Efforts have been made to use blood biomarkers to predict treatment response \cite{krintel_prediction_2015}. There has been a particular focus on identifying genetic predictors of response to anti-TNF therapies, including evidence supporting a role of TNFA, TNFR1A, MED15, PTPRC, FcGR2A, and FcGR3A genes; though few studies thus far have been successfully replicated, with many potential markers failing in follow-up studies or meta-analyses \cite{yan_pharmacogenetics_2014,zervou_lack_2013,davila-fajardo_fcgr_2015,avila-pedretti_variation_2015,swierkot_analysis_2015}. The lack of replication may be, in part, due to nominal associations and statistical errors that increase the likelihood of false positives. In addition, previous targeted studies \textcolor{green}{investigated} candidate genes that were previously identified as RA susceptibility genes while genome-wide studies have generally been underpowered. \textcolor{red}{\st{Furthermore, most studies did not discriminate between anti-TNF agents and were performed on mixed cohorts involving a variety of TNF inhibitors.} Due to this lack of replication, attempts have been made to identify robust response phenotypes for use in pharmacogenetic studies.}

% \textcolor{orange}{*** SITE STUDY w/ STRUCTURAL PHENOTYPE...AND VECTRA SCORE?? ***}

Some groups have estimated the heritability of response to various DMARDs using these different phenotypes and multiple highly sensitive methods \cite{2013_resp_herit_abstract,plant_genetic_2014,umicevic_mirkov_estimation_2014}. An early study of 1,168 RA patients used GCTA to estimate the heritability of DAS28, SJC, TJC, and ESR using GCTA \cite{2013_resp_herit_abstract}. The results suggested that change in ESR 6 months after treatment was the most heritable phenotype (0.34) while the change in TJC had the lowest estimate of heritability (0.05). The same group published a review in 2014 that presented heritability estimates from a cohort that had been treated specifically with anti-TNF monoclonal antibodies. These results suggested that SJC was the most heritable outcome while global health assessment was the least heritable \cite{plant_genetic_2014}. Most recently, two methods were used to estimate the heritability of response to anti-TNF agents in a cohort of 878 patients 14 weeks after treatment. While the methods were consistent with one another, the results conflicted with previous studies, showing that SJC had the highest heritability estimate (Bayz: 0.44-0.97; GCTA: 0.87$\pm$0.07) while ESR and VASGH had the lowest estimates (Bayz/ESR: 0.00-0.44; GCTA/ESR: 0.00$\pm$0.33; Bayz/VASGH: 0.00-0.66; GCTA/VASGH: 0.00$\pm$0.38) \cite{umicevic_mirkov_estimation_2014}.

Some statistical issues exist, however, that may limit the utility of these studies. First, some of the phenotypes used to quantify drug response violate the basic assumptions of common tests of association. Violation of such assumptions draws into question the significance of any results stemming from these tests as the test statistic may not follow the expected null distribution. In addition, most studies look at a response measurement at a single time point after the drug is given. Because the response measurements can vary over repeated measurements, considering disease state over multiple time points could increase one's power to detect an association. Furthermore, most studies did not account for placebo response when interpretating their results.

Here, we expand on the results of previous studies and address some of their shortcomings. First, we investigate several possible phenotypes, accounting for common covariates, and assess how well they conform to the assumptions of common association methods. Second, we show that some phenotypes are more or less robust at measuring the true efficacy of a drug rather than a patient’s placebo response to treatment. Third, we show that using the change in a response measure at a single time point is inconsistent and likely to result in a loss of statistical power to detect a true effect. Fourth, we propose several derived statistics based on repeated measurements of the phenotypes. Finally, we use estimate the heritability of these various response phenotypes and use permutation testing to validate our significant results.

% \textcolor{green}{Remove this part from the Background Section} \textcolor{red}{\st{We found that transforming the data to conform to the statistical methods and pooling multiple repeated measurements can increase the power of a study to detect an association with treatment response. In addition, we show that the change in swollen and tender joint counts had highest heritability estimates, but were most susceptible to the placebo effect, while the change in CRP levels was the least heritable but was the least affected by placebo. Overall, these results suggest that placebo responders and violations of parametric assumptions may confound the studies using SJC and TJC as response variables in pharmacogenetic studies. Furthermore, more sophisticated study designs and statistical approaches are required to parse out a patient's true response profile and make reliable predictions about the efficacy of a drug at an individual level.}}
