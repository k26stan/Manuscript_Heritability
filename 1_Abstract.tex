%% ABSTRACT %%

Several different classes of drug have been approved to treat rheumatoid arthritis; however, patients respond differently to different treatments. Pharmacogenetic studies have attempted to estimate the contributions of genetic variants to the variability in response to these treatments. Drug response may be defined by many distinct metrics, and identifying the most robust phenotype may increase reproducibility of these studies. Previous studies have attempted to estimate the heritability of various response phenotypes, including disease activity score, various markers of inflammation, and swollen and tender joint counts, in an effort to determine the most appropriate metric to be used in these association studies. Here, we use repeated measurements to explore several possible response phenotypes and assess the feasibility of each in the context of common association tests. Our results suggest that one can increase the power of their study by considering the response phenotype over multiple repeated measures rather than relying on a single time point. Furthermore, we find that several response metrics do not conform to of common statistical methods and that transforming the data may additionally increase the power of the study. Finally, we demonstrate that, while swollen and tender joint counts show the highest heritability estimate, these results may be confounded by the increased subjectivity of these measures and an appreciable placebo response.