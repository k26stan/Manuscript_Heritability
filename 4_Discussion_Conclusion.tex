%% DISCUSSION & CONSLUSION %%

%%%%%%%%%%%%%%%%%%%%%%%%%%%%
%% DISCUSSION %%%%%%%%%%%%%%
%%%%%%%%%%%%%%%%%%%%%%%%%%%%

\section*{Discussion}

Finding pharmacogenetic predictors of response to anti-TNF agents and other DMARDs has been a goal of the RA community for years. Several studies have purported genetic associations with response, but most results have not been reproducible. These studies have often used change in DAS or a binary response variable (e.g., ACR20, ACR50, EULAR) from a single time point as a response phenotype to measure the efficacy of the treatment. Previous studies have attempted to identify biomarkers that predict treatment response and to calculate the heritability of these phenotypes, but these studies often yielded conflicting results and are not without limitations. First, binary phenotypes over-simplify a patient's nuanced response profile. Second, many studies rely on difference measurements in disease state from a single time point before and after treatment. Third, the phenotypes considered often do not conform the assumptions of the statistical methods applied, which can yield spurious results when parametric significance tests are used. Finally, these approaches do not consider the different between a true response to the treatment and the placebo effect. Here, we have addressed these limitations and attempted to identify the most appropriate phenotypes for future pharmacogenetic studies.

We first tested how several quantitative response phenotypes conform to the parametric assumptions of common association methods. We found that several response variables, including change in SJC, TJC, and CRP, violate some basic assumptions of these methods, showed that transforming the variables ameliorates these violations (Fig. \ref{FIG_2}) and results in consistently higher heritability estimates for these phenotypes (Fig. \ref{FIG_67}). Additionally, we showed that making use of repeated measurements from individual patients over time can result in a more robust phenotypes and greater estimates of heritability (Fig. \ref{FIG_67}). Because there are many factors that can affect the RA disease state at any given moment, making use of several repeated measurements can provide a more reliable estimate of treatment response. Reducing noise and accounting for unexplained variance can increase the power of one’s studies without the cost of increasing sample size. Finally, we identified a statistically significant response to active placebo and demonstrated that different response measurements show different levels of susceptibility to this placebo effect (Fig. \ref{FIG_3}). As a result, when interpreting the results of any pharmacogenetic study, one must consider how one’s results may be confounded by a more or less subjective response phenotype. In the case of RA, the commonly used SJC and TJC are more subjective measurements and are more susceptible to the placebo effect than a molecular marker of inflammation. However, because the change in CRP levels has a negligible heritability estimate, it would make sense to use a response variable that mitigates subjectivity while retaining an appreciable estimate of heritability. Future studies could also consider molecular proxies for disease activity score, such as the Vectra DA score \cite{centola_development_2013,curtis_validation_2012}.

Overall, our results are consistent with those reported previously \cite{umicevic_mirkov_estimation_2014}. In both studies, the two most heritable phenotypes were TJC and SJC while the least heritable was the change in the molecular inflammation marker (CRP or ESR). As expected, the heritability of DAS response, a composite score that includes each individual measurement, fell between the estimates for inflammation marker and joint counts. These results, however, must be considered with the fact that active placebo had a more pronounced effect on SJC and TJC than the more objective inflammation measurement, CRP. We argue that more research into the effect of placebo on these response metrics is necessary before recommending the use of SJC and TJC as an endpoint for clinical and pharmacogenetic studies. Furthermore, because of limits in our and others’ study designs, we cannot explicitly model the effect of placebo at an individual level, but rather must estimate a population-level effect. This highlights the need for sophisticated study designs for studies aiming to predict individual drug response from genetic and non-genetic factors.

%%%%%%%%%%%%%%%%%%%%%%%%%%%%
%% CONCLUSION %%%%%%%%%%%%%%
%%%%%%%%%%%%%%%%%%%%%%%%%%%%

\section*{Conclusion}

The design of pharmacogenetic studies aimed at investigating mechanisms of response is paramount. They should be designed in a way that allows researchers to distinguish placebo response from true response to the drug. Reducing the noise in these studies and using appropriate statistical approaches will increase the power to detect an effect without the cost associated with increasing the cohort size. These designs may include several crossover periods of drug and placebo treatment, repeated measurements, and more sophisticated statistical methods that account for several factors potentially affecting treatment response.

% Furthermore, making use of longitudinal study designs will allow researchers to identify more robust response profiles from patients. In order to account for placebo response, researchers should consider a crossover designs with several baseline measurements and  blocks in which patients are alternately treated with the drug and a placebo.